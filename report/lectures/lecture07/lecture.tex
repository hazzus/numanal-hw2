\documentclass[../../lectures.tex]{subfiles}

\begin{document}

\chapter{Сети}

\section{Классификация}
\begin{itemize} 
    \item \textbf{Packet switched networks}\\ Эта концепция прижилась
    \item \textbf{Circut switched networks}\\ Коммутация каналов (телефоны)\\
          Никто не может вмешаться в канал и повлиять на передачу\\
          Требует больше ресурсов.
\end{itemize}

\section{L2}
\subsection{Frame}
\centerimage{ethernet-frame.jpg}{Frame}{0.8}
\begin{itemize}
    \item \textbf{SFD} --- разграничитель
    \item \textbf{Destination MAC} -- адрес получателя (идет первым, чтобы сетевая карта не делала лишнюю работу)
    \item \textbf{Source MAC} --- адрес отправителя
    \item \textbf{Type} --- какой протокол упакован внутри
\end{itemize}

\subsection{MAC адрес}
\begin{itemize}
    \item OUI\\
          Vendor assigned\\
          (все пространство MAC-адресов закреплено за какими-то производителями)
    \item \emph{Unicast} --- least signigicat bit of the first octet is 0\\
          Передача одному
    \item \emph{Multicast} --- least significant bit of the first octet is 1\\
          Передача многим
    \item \emph{Broadcast} --- ff:ff:ff:ff:ff:ff технология, 
          Передача всем \end{itemize} 
\subsection{Broadcast domain}
\textbf{VLAN} (trunk, access) --- виртуальная локальная сеть (используется поле \textbf{type})

\subsection{Devices}
\begin{itemize}
    \item \textbf{Hub}
        \begin{itemize}
            \item Дублирует электрический сигнал
            \item "Общая шина"
            \item Коллизии при одновременной отправке
            \item Алгоритм \emph{exponential backoff} --- каждый засыпаем на произвольное количество времени при коллизии
        \end{itemize}
    \item \textbf{Switch}
        \begin{itemize}
            \item Maps MAC to port (не биективно)
            \item Active learning (изучение)
            \item Flooding (лавинная рассылка)
            \item FIB --- база
        \end{itemize}
    \item \textbf{Bridge}
\end{itemize}

\subsection{Мысленный эксперимент}
\begin{enumerate}
    \item Чтобы кто-то отправил кому-то что-то, он должен знать его MAC-адрес (допустим, знает)
    \item Машина отправляет фрейм по MAC-адресу получателя
    \item Его получает коммутатор (L2 устройство), в нем адрес отправителя и адрес получателя
    \item Добавляет адрес отправителя в базу, если не сохранен порт получателя, начинается лавинная рассылка
    \item Получатель примет фрейм с теми же адресами отправителя и получателя
    \item Получатель отправляет ответ с реверснутыми адресами
    \item У коммутатора есть все данные и он конкретно отправляет фрейм отправителю первоначального фрейма. Также запоминает порт для получателя.
\end{enumerate}

\section{L3}
\textbf{L3} --- нужен чтобы передавать данные между сетями

\subsection{IP адрес}
\begin{itemize}
    \item \emph{Unicast}
    \item \emph{Multicast} --- 224.0.0/24
    \item \emph{Broadcast} --- 255.255.255.255
\end{itemize}

\subsection{CIDR}
Сначала были классы
\begin{itemize}
    \item A-сеть --- 1 старший байт
    \item B-сеть --- 2 старших байта
    \item C-сеть --- 3 старших байта
\end{itemize}
Потом классов не стало\\
\textbf{CIDR} (Classless InterDomain Routing) --- маска сети --- /8, /16, /24\\
Весь интернет: 0.0.0.0/32\\
Половины интернета: 0.0.0.0/1, 128.0.0.0/1\\
Конкретный адрес: *.*.*.*/32

\subsection{Кто заведует адресами?}
\begin{itemize}
    \item \textbf{IANA} --- main boss
    \item \textbf{RIR} --- организации на разных континентах
    \item \textbf{LIR} --- локальные организации
    \item \shell{whois}
    \item \textbf{AS} --- автономная система (инфраструктура, находящаяся по контролем)
\end{itemize}

\subsection{Mapping}
\textbf{ARP} (Address Resolution Protocol) --- кто имеет L3-адрес такой-то, скажи мне свой L2-адрес\\
\textbf{RARP} --- \todo{?}

\subsection{Routing}
К одному и тому же месту могут быть несколько маршрутов\\
\shell{netstat -rn} --- таблица маршрутизации\\
\shell{ip r}
\begin{itemize}
    \item Статическая
    \item Динамическая
        \begin{itemize}
            \item Distance vector (Bellman-Ford)
            \item Link state (Dijkstra)
            \item IGP (RIP, OSPF)
            \item EGP --- не работал
            \item BGP --- работает (full view)
            \item stub --- только один маршрут наверх
            \item assymetric --- маршрут от отправителя к получателю необязательно совпадает с обратным маршрутом
            \item anycast --- одни и те же префиксы (в итоге получаем отказоустойчивость)
            \item ECMP (Equal Cost MultiParse)
        \end{itemize}
\end{itemize}

\subsection{Additional}
\begin{itemize}
    \item RFC1918
    \item \textbf{DHCP} --- динамическое получение IP-адреса
\end{itemize}

\subsection{Devices}
\textbf{Router}

\subsection{Мысленный эксперимент}
\todo{Картинка с доски с пояснениями}

\newpage
\section{L4}
\subsection{Ports}
\begin{itemize}
    \item Priveleged
    \item Well known (/etc/services)
    \item Ephemeral
\end{itemize}

\subsection{Protocols}
\textbf{TCP} -- reliable, connection oriented
\begin{itemize}
    \item Header
          \centerimage{ip-header.png}{IP header}{0.5}
    \item Handshaking --- рукопожатие, обмен служебными пакетами, подготовка к соединению
    \item Гарантия последовательности данных (байт №5 раньше байта №6)
    \item Границ внутри сообщений нет
    \item Внутри протокола есть сериализация (например, размер сообщений 4 гб)
    \item Пока все старые данные не придут, новые не будут приходить (хотя старые могут быть не нужны)
    \item Retransmits
    \item Congestion
    \item Timers
    \item HOL
    \item \textbf{RTT} --- Round Trip Time
    \item Active open / Passive open
\end{itemize}
\centerimage{tcp-states.png}{TCP states}{0.6}

\textbf{UDP} --- unreliable, message oriented
\begin{itemize}
    \item Данные могут быть не в том порядке
    \item Нет виртуального соединения (первоначальный шаг)
\end{itemize}

\newpage
\subsection{NAT}
\begin{itemize}
    \item Есть диапазоны сетей, которые не маршрутизируются (например, в двух компаниях одинаковые диапазоны)
    \item Позволили чуть дольше прожить с \textbf{Ipv4}
    \item \textbf{NAT} (Network Address Translation) --- технология (\textbf{L7}),  переводящая адреса, используя знания о уровнях (транспортный и т.д.) + переназначает порты
    \item Проблема с \textbf{FTP} (зашит порт)
    \item Поэтому в \textbf{NAT} добавлена функциональность deep-inspection (не очень deep)
\end{itemize}

\subsection{Разное}
\begin{itemize}
    \item \textbf{TTL} --- время жизни пакета, уменьшается на 1 при пересылке
    \item \textbf{Telnet} --- протокол, позволяющий удаленно подключаться к командной строке (данные видны всем)
    \item \textbf{SSH} --- криптография, все хорошо
    \item \shell{telnet}
    \item \shell{tcpdump} --- traffic sniffing
    \item \shell{wireshark} --- with GUI
    \item \shell{ping} --- с помощью протокола, который находится между 3 и 4 уровнями
    \item \shell{traceroute} --- путь пакета
    \item \shell{netstat -rn} --- таблица маршрутизации
    \item \shell{ip r}
\end{itemize}

\subsection{Мысленный эксперимент}
\shell{telnet vk.com 80}
\begin{itemize}
    \item Отправляем \textbf{SYN}
    \item \dots устанавливаем \textbf{TCP} соединение
    \item Различные \textbf{NAT}
\end{itemize}

\newpage
\section{L7}
\begin{itemize}
    \item Люди не машины, числа очень сложно запоминать
    \item Сначала придумали /etc/hosts (непонятно как сопровождать)
\end{itemize}

\subsection{DNS}
\begin{itemize}
    \item Иерархическая база данных
    \item \shell{dig .}
    \item Корень знает ip-адреса серверов, в которых хранятся ip-адреса следующего уровня (root servers)
    \item \textbf{NS}, \textbf{A} --- Ipv4, \textbf{AAAA} --- Ipv6, \textbf{PTR}, \textbf{MX} --- почта
\end{itemize}

\subsection{Protocols}
\begin{itemize}
    \item \textbf{QUIC} --- новомодный транспортный протокол (базируется на UDP, находится в L7, т.к. создание нового протокола в kernelspace безумно дорого)
    \item \textbf{HTTP}
    \item \textbf{HTTP2}
    \item \textbf{HTTP3}
\end{itemize}

\section{Misc}
\begin{itemize}
    \item Firewall --- сервис, фильтрующий пакеты
    \item Application Firewall
    \item \textbf{DPI}
    \item \textbf{NOC} --- узкоспециализированные сисадмины
    \item tcpdump / wireshark
\end{itemize}

\newpage
\section{Литература}
\begin{itemize}
    \item TCP/IP Illustrated volume 1 --- хорошая, старая книга
    \item \textbf{CCNA} --- примерно полугодовой курс. После сдачи сможете построить свою сеть
\end{itemize}

\section{Домашнее задание №5}
\textbf{Знакомство с сокетами}
\begin{itemize}
    \item Необходимо попробовать клиент-серверное взаимодействие через синхронные сокеты.
    \item Помимо этого нужен \textbf{Makefile}, с помощью которого можно будет собрать клиент и сервер.
    \item Семейство протоколов для использования на выбор: AF\_UNIX, AF\_INET, AF\_INET6.
\end{itemize}

\textbf{Сервер должен:}
\begin{itemize}
    \item В качестве аргументов принимать адрес, на котором будет ожидать входящих соединений
    \item Стартовать, делать \emph{bind(2)} на заданный адрес и ожидать входящих соединений
    \item При получении соединения, выполнять серверную часть придуманного вами протокола
    \item После обработки принятого соединения возвращаться в режим ожидания входящих соединений
\end{itemize}

\textbf{Клиент должен:}
\begin{itemize}
    \item Принимать параметром адрес, к которому стоит подключиться
    \item Выполнять клиентскую часть придуманного вами протокола
    \item Завершаться
\end{itemize}

Для сильных духом предлагается выбрать какой-то существующий протокол и имплементировать его, или его разумное подмножество.

Сильность духа будет оцениваться в два балла, при условии что выбранный протокол сложнее чем \textcolor{blue}{\href{https://tools.ietf.org/html/rfc862}{ECHO}}
\end{document}
