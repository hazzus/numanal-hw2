\documentclass[../../report.tex]{subfiles}

\begin{document}

\chapter{Общее}

\section{Задание}
    Исследовать различные методы численного решения систем линейных алгебраических уравнений.
    Рассмотреть: 
    \begin{itemize}
        \item Метод Гаусса с выбором ведущего элемента по остаточному стобцу и по всей остаточной матрице
        \item Метод простых итераций Якоби
        \item Метод Зайделя
        \item Метод Зайделя с последовательной релаксацией
        \item Метод спуска
    \end{itemize}

    Для каждого метода рассмотреть работу на:
    \begin{itemize}
        \item Матрице с диагональным преобладанием
        \item Случайно сгенерированной матрице
        \item Матрице Гильберта
    \end{itemize}
    Сравнить результаты.

\section{Матрицы}
    С использованием генератора случайных чисел, мы получили следующие матрицы:
\[
    M_{diag} = 
    \begin{pmatrix}
        3.848 & 0.733 & 0.335 & 0.145 & 0.535 & 0.779 & 0.28  & 0.527 & 0.036 & 0.026 \\
        0.999 & 4.822 & 0.146 & 0.155 & 0.303 & 0.634 & 0.107 & 0.641 & 0.479 & 0.654 \\
        0.88  & 0.099 & 6.259 & 0.898 & 0.584 & 0.59  & 0.201 & 0.544 & 0.369 & 0.598 \\
        0.526 & 0.485 & 0.601 & 4.827 & 0.37  & 0.36  & 0.363 & 0.353 & 0.907 & 0.094 \\
        0.226 & 0.224 & 0.81  & 0.001 & 4.058 & 0.291 & 0.905 & 0.447 & 0.424 & 0.494 \\
        0.348 & 0.049 & 0.371 & 0.06  & 0.012 & 3.95  & 0.749 & 0.416 & 0.456 & 0.476 \\
        0.877 & 0.648 & 0.738 & 0.512 & 0.614 & 0.348 & 5.427 & 0.205 & 0.42  & 0.045 \\
        0.382 & 0.143 & 0.827 & 0.89  & 0.517 & 0.829 & 0.235 & 5.631 & 0.729 & 0.807 \\
        0.687 & 0.989 & 0.549 & 0.945 & 0.024 & 0.09  & 0.624 & 0.413 & 5.239 & 0.225 \\
        0.673 & 0.502 & 0.972 & 0.195 & 0.865 & 0.661 & 0.138 & 0.857 & 0.454 & 6.748 \\

    \end{pmatrix}
\] 

Как видно из матрицы, она удовлетворяет условию диагонального преобладания \\
($|a_{ii}| \geq \sum \limits_{i \neq j} |a_{ij}| $).

\[
    M_{random} = 
    \begin{pmatrix}
        0.733 & 0.335 & 0.145 & 0.535 & 0.779 & 0.28  & 0.527 & 0.036 & 0.026 & 0.999 \\
        0.352 & 0.146 & 0.155 & 0.303 & 0.634 & 0.107 & 0.641 & 0.479 & 0.654 & 0.88  \\
        0.099 & 0.748 & 0.898 & 0.584 & 0.59  & 0.201 & 0.544 & 0.369 & 0.598 & 0.526 \\
        0.485 & 0.601 & 0.384 & 0.37  & 0.36  & 0.363 & 0.353 & 0.907 & 0.094 & 0.226 \\
        0.224 & 0.81  & 0.001 & 0.118 & 0.291 & 0.905 & 0.447 & 0.424 & 0.494 & 0.348 \\
        0.049 & 0.371 & 0.06  & 0.012 & 0.507 & 0.749 & 0.416 & 0.456 & 0.476 & 0.877 \\
        0.648 & 0.738 & 0.512 & 0.614 & 0.348 & 0.51  & 0.205 & 0.42  & 0.045 & 0.382 \\
        0.143 & 0.827 & 0.89  & 0.517 & 0.829 & 0.235 & 0.136 & 0.729 & 0.807 & 0.687 \\
        0.989 & 0.549 & 0.945 & 0.024 & 0.09  & 0.624 & 0.413 & 0.347 & 0.225 & 0.673 \\
        0.502 & 0.972 & 0.195 & 0.865 & 0.661 & 0.138 & 0.857 & 0.454 & 0.715 & 0.902 \\

    \end{pmatrix}
\]

Также приведём здесь матрицу Гильберта, несмотря на то, что её элементы полностью детерминированны.

\[
    M_{Hilbert} =
    \begin{pmatrix}
        0.5   & 0.333 & 0.25  & 0.2   & 0.167 & 0.143 & 0.125 & 0.111 & 0.1   & 0.091 \\
        0.333 & 0.25  & 0.2   & 0.167 & 0.143 & 0.125 & 0.111 & 0.1   & 0.091 & 0.083 \\
        0.25  & 0.2   & 0.167 & 0.143 & 0.125 & 0.111 & 0.1   & 0.091 & 0.083 & 0.077 \\
        0.2   & 0.167 & 0.143 & 0.125 & 0.111 & 0.1   & 0.091 & 0.083 & 0.077 & 0.071 \\
        0.167 & 0.143 & 0.125 & 0.111 & 0.1   & 0.091 & 0.083 & 0.077 & 0.071 & 0.067 \\
        0.143 & 0.125 & 0.111 & 0.1   & 0.091 & 0.083 & 0.077 & 0.071 & 0.067 & 0.062 \\
        0.125 & 0.111 & 0.1   & 0.091 & 0.083 & 0.077 & 0.071 & 0.067 & 0.062 & 0.059 \\
        0.111 & 0.1   & 0.091 & 0.083 & 0.077 & 0.071 & 0.067 & 0.062 & 0.059 & 0.056 \\
        0.1   & 0.091 & 0.083 & 0.077 & 0.071 & 0.067 & 0.062 & 0.059 & 0.056 & 0.053 \\
        0.091 & 0.083 & 0.077 & 0.071 & 0.067 & 0.062 & 0.059 & 0.056 & 0.053 & 0.05  \\
    \end{pmatrix}
\]

У всех этих матриц ненулевой определитель, то есть для любого вектора свободных членов будет существовать единственное решение. Вектор свободных членов принят единичным:
$b = \begin{pmatrix}
    1 & 1 & 1 & 1 & 1 & 1 & 1 & 1 & 1 & 1
\end{pmatrix} ^ T
$

Во всех итерационных методах принят следующий критерий прерывания работы алгоритма: $||x_i - x_{i+1}|| \leq 10^{-6}$ или $i \geq 1000$ или $||x_i - x_{i+1} \geq 10^9||$, где $||\cdot||$ является евклидовой нормой.

В качестве нормы матрицы принята норма $||\cdot||_2$, индуцированная евклидовой нормой на векторах:
$  ||A||_2 = 
   \max\limits_{||x|| = 1} ||Ax||_2 = 
   \max\limits_{\langle x, x \rangle = 1}(\sqrt{ \langle Ax, Ax \rangle}) = 
   \sqrt{\lambda_{max}(A^T A)}$
\end{document}
