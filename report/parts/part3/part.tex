\documentclass[../../report.tex]{subfiles}

\begin{document}

\chapter{Метод Зайделя}

\section{Описание}
Метод Зайделя является модификацией метода простой итерации, заключающейся в том, что при вычислении 
очередного приближения $x^{k+1}$ его уже полученные компоненты $x_1^{k+1},\dots,x_{i-1}^{k+1}$ сразу 
же используются для вычисления $x_i^{k+1}$. В общем случае данный метод подходит для численного решения 
систем линейных уравнений на матрицах с диагональным доминированием и симметричных положительно определённых матрицах. \\
Ниже приведен разбор работы алгоритма для трёх видов матриц. В каждом случе в качестве исходного вектора $x_0$ был выбран 
\[
x_0 = 
\begin{pmatrix} 
  0 \\ 
  0 \\ 
  0 \\ 
  0 \\ 
  0 \\ 
  0 \\ 
  0 \\ 
  0 \\ 
  0 \\ 
  0 
\end{pmatrix}
\]

\subsection{Матрица с диагональным доминированием}
Запуск алгоритма на подобных матрицах приведёт к 
решению независимо от выбора начального приближения. \\
После окончания итерационного процесса мы получили результат:
\[
x_k = 
\begin{pmatrix} 
  0.151843  \\ 
  0.106641  \\ 
  0.0676146 \\ 
  0.113483  \\ 
  0.160909  \\ 
  0.1879    \\ 
  0.0859165 \\ 
  0.0696215 \\ 
  0.101204  \\ 
  0.0556568 
\end{pmatrix}
\]
Такие значения переменных при умножении на матрицу дают следующий ответ
\[
\begin{pmatrix}
  0.99986 \\
  0.99996 \\
  0.99989 \\
  0.99998 \\
  1.00001 \\
  1.00005 \\
  1.00001 \\
  1.00005 \\
  1.00001 \\
  1.00000
\end{pmatrix}
\]
Этот ответ, как и ожидалось, очень близко расположен к единичному вектору свободных членов СЛАУ.

\subsection{Случайная матрица}
На случайных матрицах сходимость метода не гарантируется, 
поэтому корректная работа алгоритма на них не ожидается. Отсюда результат
\[
x_k = 
\begin{pmatrix} 
  0.24824  \\ 
 -0.614219 \\ 
  0.276654 \\ 
  0.5596   \\ 
  0.531212 \\ 
  0.964843 \\ 
  0.468153 \\ 
  0.296046 \\ 
  0.211652 \\ 
 -0.262473 
\end{pmatrix}
\]
может быть как верным, так и неверным. Проверим это умножением на матрицу, получим ответ:
\[
\begin{pmatrix}
  1.00033 \\
  0.99951 \\
  1.00015 \\
  1.00036 \\
  1.00017 \\
  0.99990 \\
  0.99931 \\
  1.00016 \\
  1.00010 \\
  1.00007
\end{pmatrix}
\]
Вывод: на данной матрице метод работает корректно.

\subsection{Матрица Гильберта}
Матрица Гильберта является симметричной положительно определённой 
матрицей, поэтому запуск алгоритма на ней приведёт к решению 
независимо от выбора начального приближения. \\
После окончания итерационного процесса мы получили результат:
\[
x_k = 
\begin{pmatrix} 
 -6.92232 \\
  39.1789 \\
 -20.8241 \\
 -30.9549 \\
 -25.7093 \\
 -30.6234 \\
  19.0445 \\
  55.8576 \\
  24.7676 \\
  4.71192
\end{pmatrix}
\]
Такие значения переменных при умножении на матрицу дают следующий ответ
\[
\begin{pmatrix}
  1.00211 \\
  0.99559 \\
  0.99418 \\
  1.00601 \\
  1.00582 \\
  1.00310 \\
  1.00060 \\
  0.99569 \\
  0.99707 \\
  0.99942
\end{pmatrix}
\]
Этот ответ, как и ожидалось, очень близко расположен к единичному вектору свободных членов СЛАУ.

\section{Исходный код}
    \code{SeidelSolver.cpp}{C++}
\end{document}
