\documentclass[../../report.tex]{subfiles}

\newcommand{\dominatedResult}{\begin{pmatrix}
    0.15188234 \\ 
    0.10664535 \\
    0.06762958 \\ 
    0.11348456 \\
    0.16090425 \\
    0.18788446 \\
    0.08590699 \\
    0.06961082 \\
    0.10119528 \\
    0.05565383
\end{pmatrix}}


\newcommand{\randomResult}{
\begin{pmatrix}
    -1.06139262 * 10^{51} \\
     1.91636147 * 10^{51} \\ 
    -2.20198358 * 10^{51} \\ 
    -1.55514353 * 10^{50} \\
    -7.15823238 * 10^{51} \\ 
     5.45658686 * 10^{51} \\
    -6.04584125 * 10^{50} \\
     1.09411930 * 10^{52} \\
    -3.06084065 * 10^{52} \\
     3.40356171 * 10^{52}
\end{pmatrix}
}

\newcommand{\hilbertResult}{
\begin{pmatrix}
     1.91416470 * 10^{44} \\
     1.46498715 * 10^{44} \\
    -8.17068353 * 10^{44} \\
     1.07334252 * 10^{45} \\
    -3.39269757 * 10^{45} \\
     4.73313389 * 10^{45} \\
    -8.02188440 * 10^{45} \\
     8.26239458 * 10^{45} \\
    -8.93384865 * 10^{45} \\
     7.76497623 * 10^{45}]
\end{pmatrix}
}


\begin{document}

\chapter{Метод Зайделя}

\section{Описание}
Метод Зайделя является модификацией метода простой итерации, заключающейся в том, что при вычислении 
очередного приближения $x^{k+1}$ его уже полученные компоненты $x_1^{k+1},\dots,x_{i-1}^{k+1}$ сразу 
же используются для вычисления $x_i^{k+1}$. В общем случае данный метод подходит для численного решения 
систем линейных уравнений на матрицах с диагональным доминированием и симметричных положительно определённых матрицах. \\
Ниже приведен разбор работы алгоритма для трёх видов матриц. В каждом случе в качестве исходного вектора $x_0$ был выбран 
\[
x_0 = 
\begin{pmatrix} 
  0 \\ 
  0 \\ 
  0 \\ 
  0 \\ 
  0 \\ 
  0 \\ 
  0 \\ 
  0 \\ 
  0 \\ 
  0 
\end{pmatrix}
\]

\subsection{Матрица с диагональным доминированием}
Запуск алгоритма на подобных матрицах приведёт к 
решению независимо от выбора начального приближения. \\
При этом выполнено условие $||B|| < 1$, следовательно справедлива апостериорная оценка погрешности. Полученный поправочный множитель равен 0,671487427, таким образом, критерием окончания итерационного процесса является выполнение следующего условия:
$$||x^{k+1} - x^k|| < 0,671487427\epsilon$$
После окончания итерационного процесса мы получили результат:
\[
x_k = \dominatedResult
\]
Такие значения переменных при умножении на матрицу дают следующий ответ
\[
\begin{pmatrix}
  0.99986 \\
  0.99996 \\
  0.99989 \\
  0.99998 \\
  1.00001 \\
  1.00005 \\
  1.00001 \\
  1.00005 \\
  1.00001 \\
  1.00000
\end{pmatrix}
\]
Этот ответ, как и ожидалось, очень близко расположен к единичному вектору свободных членов СЛАУ.

\subsection{Случайная матрица}
На случайных матрицах сходимость метода не гарантируется, 
поэтому корректная работа алгоритма на них не ожидается. Отсюда результат
\[
x_k = \randomResult
\]
может быть как верным, так и неверным. При этом условие $||B|| < 1$ не выполнено, следовательно нельзя получить апостериорную оценку погрешности. Проверим результат умножением на матрицу, получим ответ:
\[
\begin{pmatrix}
  1.00033 \\
  0.99951 \\
  1.00015 \\
  1.00036 \\
  1.00017 \\
  0.99990 \\
  0.99931 \\
  1.00016 \\
  1.00010 \\
  1.00007
\end{pmatrix}
\]
Вывод: на данной матрице метод работает корректно.

\subsection{Матрица Гильберта}
Матрица Гильберта является симметричной положительно определённой 
матрицей, поэтому запуск алгоритма на ней приведёт к решению 
независимо от выбора начального приближения. \\
При этом условие $||B|| < 1$ не выполнено, следовательно нельзя получить апостериорную оценку погрешности. \\
После окончания итерационного процесса мы получили результат:
\[
x_k = \hilbertResult
\]
Такие значения переменных при умножении на матрицу дают следующий ответ
\[
\begin{pmatrix}
  0.99999 \\
  1.00000 \\
  0.99999 \\
  1.00000 \\
  1.00000 \\
  0.99999 \\
  1.00000 \\
  1.00000 \\
  0.99999 \\
  1.00000
\end{pmatrix}
\]
Этот ответ, как и ожидалось, очень близко расположен к единичному вектору свободных членов СЛАУ.

\section{Исходный код}
    \code{SeidelSolver.cpp}{C++}
\end{document}
