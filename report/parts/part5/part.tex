\documentclass[../../report.tex]{subfiles}

\begin{document}

\chapter{Метод спуска (метод сопряженых градиентов)}

\section{Описание}
\todo{Тут надо описание}\\
\todo{Надо ли вставлять Iterations convergence?}\\
\todo{Немного слов о кажой матрице?}

В качестве исходного вектора $x_0$ был выбран
\[
x_0 =
\begin{pmatrix}
  1 \\
  1 \\
  1 \\
  1 \\
  1 \\
  1 \\
  1 \\
  1 \\
  1 \\
  1 \\
\end{pmatrix}
\]

\subsection{Матрица с диагональным доминированием}
После окончания итерационного процесса мы получили результат:
\[
x_{k} = 
\begin{pmatrix}
    0.151857307 \\
    0.105069509 \\
    0.066953175 \\
    0.113722094 \\
    0.162132957 \\
    0.189063905 \\
    0.085577424 \\
    0.069391437 \\
    0.101623562 \\
    0.054856058 \\
\end{pmatrix}
\]

Такие значения переменных при умножении на матрицу дают следующий ответ:
\[
\begin{pmatrix}
    0.999919471 \\
    0.992935755 \\
    0.996705047 \\
    1.000952084 \\
    1.003807075 \\
    1.003821879 \\
    0.998049274 \\
    0.999380378 \\
    1.000178760 \\
    0.994999328 \\
\end{pmatrix}
\]

\subsection{Случайная матрица}
После окончания итерационного процесса мы получили результат:
\[
x_{k} = 
\begin{pmatrix}
    0.265145649 \\
  -0.5803204202 \\
    0.272230533 \\
    0.501012356 \\
    0.584730686 \\
    0.940385886 \\
    0.481946997 \\
    0.284335484 \\
    0.204016369 \\
   -0.289827712 \\
\end{pmatrix}
\]

Такие значения переменных при умножении на матрицу дают следующий ответ:
\[
\begin{pmatrix}
    1.006261397 \\
    0.997450549 \\
    0.999884019 \\
    1.003293338 \\
    1.005843226 \\
    0.993926547 \\
    0.990311199 \\
    1.003386756 \\
    1.000899997 \\
    0.998336539 \\
\end{pmatrix}
\]

\subsection{Матрица Гильберта}
После окончания итерационного процесса мы получили результат:
\[
x_{k} = 
\begin{pmatrix}
    -5.497880066 \\
    25.104304276 \\
     1.876343633 \\
   -20.958817049 \\
   -37.426014627 \\
   -42.847560126 \\
     6.213352584 \\
    49.579013029 \\
    38.083103312 \\
    15.322952107 \\
\end{pmatrix}
\]

Такие значения переменных при умножении на матрицу дают следующий ответ:
\[
\begin{pmatrix}
    0.993408741 \\
    0.997514045 \\
    1.002088716 \\
    1.003065894 \\
    1.001984762 \\
    1.002247829 \\
    1.002251921 \\
    1.002533145 \\
    1.003737600 \\
    1.003225523 \\
\end{pmatrix}
\]

\section{Исходный код}
\code{src/system/solver/Solver.java}{Java}

\end{document}
