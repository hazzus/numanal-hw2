\documentclass[../../report.tex]{subfiles}

\newcommand{\dominatedResult}{
\begin{pmatrix} 
0.151881422 \\
0.106646453 \\
0.0676301244 \\
0.113485461 \\
0.160905627 \\
0.187885977 \\
0.0859075715 \\
0.0696112425 \\
0.10119529 \\
0.055653586
\end{pmatrix}}

\newcommand{\randomResult}{
\begin{pmatrix} 
0.248430651 \\ 
-0.617896379 \\
0.27675287 \\
0.564193533 \\
0.528634855 \\
0.967365431 \\
0.46685821 \\
0.29630525 \\
0.213863252 \\
-0.262269939
\end{pmatrix}
}

\newcommand{\hilbertResult}{
\begin{pmatrix} 
-6.2624771 \\
31.0389149 \\
-8.94248913 \\
-16.873137 \\
-35.113627 \\
-46.62674 \\
11.4857659 \\
52.0688052 \\
33.0079033 \\
15.6151721
\end{pmatrix}
}


\begin{document}

\chapter{Метод спуска (метод сопряженых градиентов)}

\section{Описание}
\begin{itemize}
  \item Методы спуска численного решения СЛАУ предполагают построение функции $F(x)=F(x_1, x_2, \dots, x_n)$, 
имеющей гладкий минимум в точке решения данной СЛАУ, и нахождение этого
минимума каким-либо методом оптимизации.

  \item Нахождение минимума сводится к последовательному выбору направления и шага спуска, таких, что на каждой итерации значение
функции в очередной точке будет меньше предыдущего. Итерации продолжаются до достижения
какого-либо условия окончания, обеспечивающего требуемую точность.

  \item В методе сопряженных градиентов в качестве направлений спуска последовательно берутся
вектора $p_0, p_1, \dots, p_{n-1}$, взаимно сопряженные относительно матрицы $A$. При точном выполнении
операций метод сходится к точному решению не более, чем за n шагов. 

  \item Однако, на практике, округления при выполнении арифметических операций приводят к неточности в решении СЛАУ.
Данный метод обеспечивает высокую скорость сходимости.
\end{itemize}

В качестве начального вектора для всех трех матриц возьмем вектор $x_0$
\[
x_0 =
\begin{pmatrix}
  1 \\
  1 \\
  1 \\
  1 \\
  1 \\
  1 \\
  1 \\
  1 \\
  1 \\
  1 \\
\end{pmatrix}
\]

\subsection{Матрица с диагональным доминированием}
Данная матрица не является симметричной, поэтому нет гарантии, что метод приведет к верному
решению. Проверим, удастся ли получить верное решение методом сопряженных градиентов.

После окончания итерационного процесса мы получили результат:
\[
x_{k} = \dominatedResult
\]

Такие значения переменных при умножении на матрицу дают следующий ответ:
\[
\begin{pmatrix}
    0.999919471 \\
    0.992935755 \\
    0.996705047 \\
    1.000952084 \\
    1.003807075 \\
    1.003821879 \\
    0.998049274 \\
    0.999380378 \\
    1.000178760 \\
    0.994999328 \\
\end{pmatrix}
\]

Как можно увидеть, полученное решение успешно проходит проверку умножением на матрицу $A$
и сравнением со столбцом свободных членов нашей СЛАУ.

\subsection{Случайная матрица}
Данная матрица не является симметричной, но с помощью домножения обеих частей СЛАУ на транспонированную матрицу
мы получим систему, удовлетворяющую требованиям метода сопряженных градиентов.

После окончания итерационного процесса мы получили результат:
\[
x_{k} = \randomResult
\]

Такие значения переменных при умножении на матрицу дают следующий ответ:
\[
\begin{pmatrix}
    1.006261397 \\
    0.997450549 \\
    0.999884019 \\
    1.003293338 \\
    1.005843226 \\
    0.993926547 \\
    0.990311199 \\
    1.003386756 \\
    1.000899997 \\
    0.998336539 \\
\end{pmatrix}
\]
Получаем решение, достаточно хорошо проходящее проверку.

\subsection{Матрица Гильберта}
Данная матрица удовлетворяет условиям метода сопряженных градиентов, поэтому применим
метод для решения данной СЛАУ.

После окончания итерационного процесса мы получили результат:
\[
x_{k} = \hilbertResult
\]

Такие значения переменных при умножении на матрицу дают следующий ответ:
\[
\begin{pmatrix}
    0.993408741 \\
    0.997514045 \\
    1.002088716 \\
    1.003065894 \\
    1.001984762 \\
    1.002247829 \\
    1.002251921 \\
    1.002533145 \\
    1.003737600 \\
    1.003225523 \\
\end{pmatrix}
\]

Снова получаем решение, достаточно хорошо проходящее проверку.

\section{Исходный код}
\code{src/system/solver/Solver.java}{Java}

\end{document}
