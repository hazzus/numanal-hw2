\documentclass[../../report.tex]{subfiles}

\newcommand{\dominatedResult}{\begin{pmatrix}
    0.15188234 \\ 
    0.10664535 \\
    0.06762958 \\ 
    0.11348456 \\
    0.16090425 \\
    0.18788446 \\
    0.08590699 \\
    0.06961082 \\
    0.10119528 \\
    0.05565383
\end{pmatrix}}


\newcommand{\randomResult}{
\begin{pmatrix}
    -1.06139262 * 10^{51} \\
     1.91636147 * 10^{51} \\ 
    -2.20198358 * 10^{51} \\ 
    -1.55514353 * 10^{50} \\
    -7.15823238 * 10^{51} \\ 
     5.45658686 * 10^{51} \\
    -6.04584125 * 10^{50} \\
     1.09411930 * 10^{52} \\
    -3.06084065 * 10^{52} \\
     3.40356171 * 10^{52}
\end{pmatrix}
}

\newcommand{\hilbertResult}{
\begin{pmatrix}
     1.91416470 * 10^{44} \\
     1.46498715 * 10^{44} \\
    -8.17068353 * 10^{44} \\
     1.07334252 * 10^{45} \\
    -3.39269757 * 10^{45} \\
     4.73313389 * 10^{45} \\
    -8.02188440 * 10^{45} \\
     8.26239458 * 10^{45} \\
    -8.93384865 * 10^{45} \\
     7.76497623 * 10^{45}]
\end{pmatrix}
}


\begin{document}

\chapter{Метод спуска (метод сопряженых градиентов)}

\section{Описание}
\begin{itemize}
  \item Методы спуска численного решения СЛАУ предполагают построение функции $F(x)=F(x_1, x_2, \dots, x_n)$, 
имеющей гладкий минимум в точке решения данной СЛАУ, и нахождение этого
минимума каким-либо методом оптимизации.

  \item Нахождение минимума сводится к последовательному выбору направления и шага спуска, таких, что на каждой итерации значение
функции в очередной точке будет меньше предыдущего. Итерации продолжаются до достижения
какого-либо условия окончания, обеспечивающего требуемую точность.

  \item В методе сопряженных градиентов в качестве направлений спуска последовательно берутся
вектора $p_0, p_1, \dots, p_{n-1}$, взаимно сопряженные относительно матрицы $A$. При точном выполнении
операций метод сходится к точному решению не более, чем за n шагов. 

  \item Однако, на практике, округления при выполнении арифметических операций приводят к неточности в решении СЛАУ.
Данный метод обеспечивает высокую скорость сходимости.
\end{itemize}

В качестве начального вектора для всех трех матриц возьмем вектор $x_0$
\[
x_0 =
\begin{pmatrix}
  1 \\
  1 \\
  1 \\
  1 \\
  1 \\
  1 \\
  1 \\
  1 \\
  1 \\
  1 \\
\end{pmatrix}
\]

\subsection{Матрица с диагональным доминированием}
Данная матрица не является симметричной, поэтому, перед применением метода сопряженных градиентов умножим
обе части системы на транспонированную матрицу.
После этого матрица системы будет удовлетворять требованиям метода.

После окончания итерационного процесса мы получили результат:
\[
x_{k} = \dominatedResult
\]

Такие значения переменных при умножении на матрицу дают следующий ответ:
\[
\begin{pmatrix}
    0.9999999999999967 \\
    1.0000000000000018 \\
    0.9999999999999983 \\
    1.0000000000000044 \\
    1.0000000000000018 \\
    1.000000000000001 \\
    0.9999999999999999 \\
    1.0 \\
    1.0000000000000002 \\
    1.0000000000000024 \\
\end{pmatrix}
\]

Как можно увидеть, полученное решение успешно проходит проверку умножением на матрицу $A$
и сравнением со столбцом свободных членов нашей СЛАУ.

\subsection{Случайная матрица}
Данная матрица не является симметричной, но с помощью домножения обеих частей СЛАУ на транспонированную матрицу
мы получим систему, удовлетворяющую требованиям метода сопряженных градиентов.

После окончания итерационного процесса мы получили результат:
\[
x_{k} = \randomResult
\]

Такие значения переменных при умножении на матрицу дают следующий ответ:
\[
\begin{pmatrix}
    0.9999999999999225 \\
    0.9999999999999547 \\
    0.9999999999999911 \\
    0.9999999999999395 \\
    0.999999999999895 \\
    0.9999999999998908 \\
    0.9999999999999512 \\
    0.9999999999999876 \\
    0.9999999999999094 \\
    0.9999999999999184 \\
\end{pmatrix}
\]
Получаем решение, достаточно хорошо проходящее проверку.

\subsection{Матрица Гильберта}
Данная матрица удовлетворяет условиям метода сопряженных градиентов, поэтому применим
метод для решения данной СЛАУ.

После окончания итерационного процесса мы получили результат:
\[
x_{k} = \hilbertResult
\]

Такие значения переменных при умножении на матрицу дают следующий ответ:
\[
\begin{pmatrix}
    0.9999999999999896 \\
    0.9999999999999831 \\
    1.0000000000000335 \\
    0.9999999999999338 \\
    1.0000000000000875 \\
    0.9999999999998848 \\
    1.0000000000001474 \\
    0.9999999999998788 \\
    1.0000000000000608 \\
    0.9999999999999741 \\
\end{pmatrix}
\]

Снова получаем решение, достаточно хорошо проходящее проверку.

\section{Исходный код}
\code{src/system/solver/Solver.java}{Java}

\end{document}
