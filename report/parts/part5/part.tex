\documentclass[../../report.tex]{subfiles}

\begin{document}

\chapter{Метод спуска (метод сопряженых градиентов)}
\todo{Iterations convergence?}\\
\todo{Немного слов о кажой матрице?}\\

\section{Описание}
\begin{itemize}
  \item Методы спуска численного решения СЛАУ предполагают построение функции $F(x)=F(x_1, x_2, \dots, x_n)$, 
имеющей гладкий минимум в точке решения данной СЛАУ, и нахождение этого
минимума каким-либо методом оптимизации.

  \item Нахождение минимума сводится к последовательному выбору направления и шага спуска, таких, что на каждой итерации значение
функции в очередной точке будет меньше предыдущего. Итерации продолжаются до достижения
какого-либо условия окончания, обеспечивающего требуемую точность.

  \item В методе сопряженных градиентов в качестве направлений спуска последовательно берутся
вектора $p_0, p_1, \dots, p_{n-1}$, взаимно сопряженные относительно матрицы $A$. При точном выполнении
операций метод сходится к точному решению не более, чем за n шагов. 

  \item Однако, на практике, округления при выполнении арифметических операций приводят к неточности в решении СЛАУ.
Данный метод обеспечивает высокую скорость сходимости.
\end{itemize}

В качестве начального вектора для всех трех матриц возьмем вектор $x_0$
\[
x_0 =
\begin{pmatrix}
  1 \\
  1 \\
  1 \\
  1 \\
  1 \\
  1 \\
  1 \\
  1 \\
  1 \\
  1 \\
\end{pmatrix}
\]

\subsection{Матрица с диагональным доминированием}
После окончания итерационного процесса мы получили результат:
\[
x_{k} = 
\begin{pmatrix}
    0.151857307 \\
    0.105069509 \\
    0.066953175 \\
    0.113722094 \\
    0.162132957 \\
    0.189063905 \\
    0.085577424 \\
    0.069391437 \\
    0.101623562 \\
    0.054856058 \\
\end{pmatrix}
\]

Такие значения переменных при умножении на матрицу дают следующий ответ:
\[
\begin{pmatrix}
    0.999919471 \\
    0.992935755 \\
    0.996705047 \\
    1.000952084 \\
    1.003807075 \\
    1.003821879 \\
    0.998049274 \\
    0.999380378 \\
    1.000178760 \\
    0.994999328 \\
\end{pmatrix}
\]

\subsection{Случайная матрица}
После окончания итерационного процесса мы получили результат:
\[
x_{k} = 
\begin{pmatrix}
    0.265145649 \\
  -0.5803204202 \\
    0.272230533 \\
    0.501012356 \\
    0.584730686 \\
    0.940385886 \\
    0.481946997 \\
    0.284335484 \\
    0.204016369 \\
   -0.289827712 \\
\end{pmatrix}
\]

Такие значения переменных при умножении на матрицу дают следующий ответ:
\[
\begin{pmatrix}
    1.006261397 \\
    0.997450549 \\
    0.999884019 \\
    1.003293338 \\
    1.005843226 \\
    0.993926547 \\
    0.990311199 \\
    1.003386756 \\
    1.000899997 \\
    0.998336539 \\
\end{pmatrix}
\]

\subsection{Матрица Гильберта}
После окончания итерационного процесса мы получили результат:
\[
x_{k} = 
\begin{pmatrix}
    -5.497880066 \\
    25.104304276 \\
     1.876343633 \\
   -20.958817049 \\
   -37.426014627 \\
   -42.847560126 \\
     6.213352584 \\
    49.579013029 \\
    38.083103312 \\
    15.322952107 \\
\end{pmatrix}
\]

Такие значения переменных при умножении на матрицу дают следующий ответ:
\[
\begin{pmatrix}
    0.993408741 \\
    0.997514045 \\
    1.002088716 \\
    1.003065894 \\
    1.001984762 \\
    1.002247829 \\
    1.002251921 \\
    1.002533145 \\
    1.003737600 \\
    1.003225523 \\
\end{pmatrix}
\]

\section{Исходный код}
\code{src/system/solver/Solver.java}{Java}

\end{document}
