\documentclass[../../report.tex]{subfiles}

\newcommand{\dominatedResult}{\begin{pmatrix}
    0.15188234 \\ 
    0.10664535 \\
    0.06762958 \\ 
    0.11348456 \\
    0.16090425 \\
    0.18788446 \\
    0.08590699 \\
    0.06961082 \\
    0.10119528 \\
    0.05565383
\end{pmatrix}}


\newcommand{\randomResult}{
\begin{pmatrix}
    -1.06139262 * 10^{51} \\
     1.91636147 * 10^{51} \\ 
    -2.20198358 * 10^{51} \\ 
    -1.55514353 * 10^{50} \\
    -7.15823238 * 10^{51} \\ 
     5.45658686 * 10^{51} \\
    -6.04584125 * 10^{50} \\
     1.09411930 * 10^{52} \\
    -3.06084065 * 10^{52} \\
     3.40356171 * 10^{52}
\end{pmatrix}
}

\newcommand{\hilbertResult}{
\begin{pmatrix}
     1.91416470 * 10^{44} \\
     1.46498715 * 10^{44} \\
    -8.17068353 * 10^{44} \\
     1.07334252 * 10^{45} \\
    -3.39269757 * 10^{45} \\
     4.73313389 * 10^{45} \\
    -8.02188440 * 10^{45} \\
     8.26239458 * 10^{45} \\
    -8.93384865 * 10^{45} \\
     7.76497623 * 10^{45}]
\end{pmatrix}
}


\begin{document}

\chapter{Метод спуска (метод сопряженых градиентов)}
\todo{Iterations convergence?}\\
\todo{Немного слов о кажой матрице?}\\

\section{Описание}
\begin{itemize}
  \item Методы спуска численного решения СЛАУ предполагают построение функции $F(x)=F(x_1, x_2, \dots, x_n)$, 
имеющей гладкий минимум в точке решения данной СЛАУ, и нахождение этого
минимума каким-либо методом оптимизации.

  \item Нахождение минимума сводится к последовательному выбору направления и шага спуска, таких, что на каждой итерации значение
функции в очередной точке будет меньше предыдущего. Итерации продолжаются до достижения
какого-либо условия окончания, обеспечивающего требуемую точность.

  \item В методе сопряженных градиентов в качестве направлений спуска последовательно берутся
вектора $p_0, p_1, \dots, p_{n-1}$, взаимно сопряженные относительно матрицы $A$. При точном выполнении
операций метод сходится к точному решению не более, чем за n шагов. 

  \item Однако, на практике, округления при выполнении арифметических операций приводят к неточности в решении СЛАУ.
Данный метод обеспечивает высокую скорость сходимости.
\end{itemize}

В качестве начального вектора для всех трех матриц возьмем вектор $x_0$
\[
x_0 =
\begin{pmatrix}
  1 \\
  1 \\
  1 \\
  1 \\
  1 \\
  1 \\
  1 \\
  1 \\
  1 \\
  1 \\
\end{pmatrix}
\]

\subsection{Матрица с диагональным доминированием}
После окончания итерационного процесса мы получили результат:
\[
x_{k} = \dominatedResult
\]

Такие значения переменных при умножении на матрицу дают следующий ответ:
\[
\begin{pmatrix}
    0.999919471 \\
    0.992935755 \\
    0.996705047 \\
    1.000952084 \\
    1.003807075 \\
    1.003821879 \\
    0.998049274 \\
    0.999380378 \\
    1.000178760 \\
    0.994999328 \\
\end{pmatrix}
\]

\subsection{Случайная матрица}
После окончания итерационного процесса мы получили результат:
\[
x_{k} = \randomResult
\]

Такие значения переменных при умножении на матрицу дают следующий ответ:
\[
\begin{pmatrix}
    1.006261397 \\
    0.997450549 \\
    0.999884019 \\
    1.003293338 \\
    1.005843226 \\
    0.993926547 \\
    0.990311199 \\
    1.003386756 \\
    1.000899997 \\
    0.998336539 \\
\end{pmatrix}
\]

\subsection{Матрица Гильберта}
После окончания итерационного процесса мы получили результат:
\[
x_{k} = \hilbertResult
\]

Такие значения переменных при умножении на матрицу дают следующий ответ:
\[
\begin{pmatrix}
    0.993408741 \\
    0.997514045 \\
    1.002088716 \\
    1.003065894 \\
    1.001984762 \\
    1.002247829 \\
    1.002251921 \\
    1.002533145 \\
    1.003737600 \\
    1.003225523 \\
\end{pmatrix}
\]

\section{Исходный код}
\code{src/system/solver/Solver.java}{Java}

\end{document}
