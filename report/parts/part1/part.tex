\documentclass[../../report.tex]{subfiles}

\begin{document}

\newcommand{\dominatedResult}{
\begin{pmatrix} 
0.151881422 \\
0.106646453 \\
0.0676301244 \\
0.113485461 \\
0.160905627 \\
0.187885977 \\
0.0859075715 \\
0.0696112425 \\
0.10119529 \\
0.055653586
\end{pmatrix}}

\newcommand{\randomResult}{
\begin{pmatrix} 
0.248430651 \\ 
-0.617896379 \\
0.27675287 \\
0.564193533 \\
0.528634855 \\
0.967365431 \\
0.46685821 \\
0.29630525 \\
0.213863252 \\
-0.262269939
\end{pmatrix}
}

\newcommand{\hilbertResult}{
\begin{pmatrix} 
-6.2624771 \\
31.0389149 \\
-8.94248913 \\
-16.873137 \\
-35.113627 \\
-46.62674 \\
11.4857659 \\
52.0688052 \\
33.0079033 \\
15.6151721
\end{pmatrix}
}


\chapter{Метод Гаусса}

В стандартном методе Гаусса мы решаем систему уравнений вида: $A \cdot x = B$.
Где $A \cdot x = B$ это
$\sum \limits_{j = 1}^{n} a_{i,j} \cdot x_j = b_i$ Для $1 \le i \le n$.
Обычный способ решения~--- приведение исходной матрицы к диагональному виду.

При простом методе решения возникает проблема, связанная с тем, что числа хранятся в double. Например, если на диагонали в строчке $i$ стоит величина $a_{i,i}$ близкая к $0$, то мы будем домножать следующую строчку на число $\frac{a_{i+1,i}}{a_{i,i}}$. При этом, возникает большая погрешность при делении на число, близкое к $0$.

Чтобы избежать этого, можно усовершенствовать стандартный алгоритм и сделать метод Гаусса с выбором ведущего элемента по остаточному столбцу или метод Гаусса с ведущего элемента по всей остаточной матрице.
Стандартный метод Гаусса не работает на некоторых матрицах, например, если $a_{1,1} = 0$.
Число операций, производимых алгоритмом $\approx \frac{2}{3} \cdot n^3$.

\section{Метод Гаусса с выбором ведущего элемента по остаточному столбцу}
Модификация по сравнению с обычным методом состоит в том, что мы можем менять порядок строк и выбирать для текущей строки ту, у которой ведущий элемент максимальный по модулю из оставшихся строк.
При такой модификации у нас повышается устойчивость.

Алгоритм работает на всех матрицах, имеющих решение, то есть на тех, у которых $det(A) \neq 0$.

Число операций, производимых алгоритмом $\approx \frac{2}{3} \cdot n^3$, что эквивалентно стандартному методу Гаусса.

\section{Метод Гаусса с выбором ведущего элемента по всей остаточной матрице}
Если в методе Гаусса с выбором ведущего элемента по остаточному столбцу погрешность продолжает расти, то можно воспользоваться методом с выбором ведущего элемента по всей остаточной матрице.

Модификация по сравнению с обычным методом состоит в том, что мы можем менять порядок строк и столбцов. При обработки {k}-ой строки мы выбираем элемент из оставшейся необработанной матрицы $a_{l,h}= \max \limits_{k \le i \le n;~k \le j \le n} a_{i,j}$, меняем местами строки $k$ и $l$ и столбцы $k$ и $h$.

При такой модификации у нас повышается устойчивость.

Алгоритм работает на всех матрицах, имеющих решение, то есть на тех, у которых $det(A) \neq 0$.

Число операций, производимых алгоритмом $n^3$, что больше, чем в предыдущих двух методах в $1.5$ раза.

\section{Примеры работ на некоторых матрицах}
Ответы на следующие матрицы получились полностью одинаковыми для метода Гаусса с выбором ведущего элемента по остаточному столбцу и для метода Гаусса с выбором ведущего элемента по всей остаточной матрице.

\subsection{Матрица с диагональным доминированием}
Ответ для данной матрицы:
\[
x = \dominatedResult
\]
Такие значения переменных при умножении на исходную матрицу дают следующий ответ:
\[
\begin{pmatrix}
1 \\
1 \\
1 \\
1 \\
1 \\
1 \\
1 \\
1 \\
1 \\
1
\end{pmatrix}
\]
Этот вектор совпадает с единичном вектором свободных членов СЛАУ $= B$, поэтому можно сделать вывод, что методы Гаусса с выбором ведущего элемента по остаточному столбцу и по всей остаточной матрице работают корректно на матрицах с диагональным доминированием.

\subsection{Матрица Гильберта}
Ответ для данной матрицы:
\[
x = \hilbertResult
\]
Такие значения переменных при умножении на исходную матрицу дают следующий ответ:
\[
\begin{pmatrix}
1 \\
1 \\
1 \\
1 \\
1 \\
1 \\
1 \\
1 \\
1 \\
1
\end{pmatrix}
\]
Этот вектор совпадает с единичном вектором свободных членов СЛАУ $= B$, поэтому можно сделать вывод, что методы Гаусса с выбором ведущего элемента по остаточному столбцу и по всей остаточной матрице работают корректно на матрице Гильберта.

\subsection{Случайная матрица}
Ответ для данной матрицы:
\[
x = \randomResult
\]

Такие значения переменных при умножении на исходную матрицу дают следующий ответ:
\[
\begin{pmatrix}
1 \\
1 \\
1 \\
1 \\
1 \\
1 \\
1 \\
1 \\
1 \\
1
\end{pmatrix}
\]
Этот вектор совпадает с единичном вектором свободных членов СЛАУ $= B$, поэтому можно сделать вывод, что методы Гаусса с выбором ведущего элемента по остаточному столбцу и по всей остаточной матрице работают корректно на случайных матрицах.

\section{Исходный код}
    \code{gaussian-matrix.h}{C++}
    \code{gaussian-column.h}{C++}
\end{document}
