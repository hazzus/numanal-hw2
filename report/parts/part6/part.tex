\documentclass[../../report.tex]{subfiles}

\begin{document}

\chapter{Результаты}

\section{Результаты расчетов}
    Приведём здесь записанные подряд векторы, которые были получены в результате работы метода на каждой матрице. Вот они, слева направо: Намджун, Чонгук, Чингачгук, Гойко Митич, Джин, Юнги.
    Также сравним евклидову норму разности вектора свободных членов и результата работы алгоритма.
\subsection{Матрица с диагональным преобладанием}
\[
    Jacobi:
    \begin{pmatrix}
        0.154703 \\
        0.109403 \\
        0.070100 \\
        0.116197 \\
        0.163861 \\
        0.190267 \\
        0.088534 \\
        0.072591 \\
        0.103984 \\
        0.058208
    \end{pmatrix}
\]
Норма разности для метода Якоби: $0.0065623$.
\section{Графики}
    Какие-нибудь графики
    
\end{document}
