\documentclass[../../report.tex]{subfiles}

\newcommand{\dominatedResult}{
\begin{pmatrix} 
0.151881422 \\
0.106646453 \\
0.0676301244 \\
0.113485461 \\
0.160905627 \\
0.187885977 \\
0.0859075715 \\
0.0696112425 \\
0.10119529 \\
0.055653586
\end{pmatrix}}

\newcommand{\randomResult}{
\begin{pmatrix} 
0.248430651 \\ 
-0.617896379 \\
0.27675287 \\
0.564193533 \\
0.528634855 \\
0.967365431 \\
0.46685821 \\
0.29630525 \\
0.213863252 \\
-0.262269939
\end{pmatrix}
}

\newcommand{\hilbertResult}{
\begin{pmatrix} 
-6.2624771 \\
31.0389149 \\
-8.94248913 \\
-16.873137 \\
-35.113627 \\
-46.62674 \\
11.4857659 \\
52.0688052 \\
33.0079033 \\
15.6151721
\end{pmatrix}
}


\begin{document}

\chapter[Метод Зайделя с последовательной релаксацией]{\texorpdfstring{Метод Зайделя \\ 
с последовательной релаксацией}{Метод Зайделя с последовательной релаксацией}}

\section{Описание}
Метод Зейделя с последовательной релаксацией это вариант метода Зейделя для решения линейных систем уравнений с подбором отимального коэфициента релаксации. Позволяет добиться  более быстрой сходимости. \\
Исходной матрицей был выбран вектор $x_0$ 
\[
x_0 = \begin{pmatrix}
    0 \\ 
    0 \\ 
    0 \\ 
    0 \\ 
    0 \\ 
    0 \\ 
    0 \\ 
    0 \\ 
    0 \\ 
    0 
\end{pmatrix}
\]

\subsection{Матрица с диагональным доминированием}
После проведения $n=30$ итераций, был получен результат
\[
x_{30} = \dominatedResult
\]

Такие значения переменных при умножении на матрицу дают следующий ответ
\[
\begin{pmatrix}
    1.00000011346 \\
    0.99999382372 \\
    0.99999457808 \\
    0.9999938682 \\
    0.99999288919 \\
    0.99999349964 \\
    0.9999946181 \\
    0.99999464458 \\
    0.99999768484 \\
    0.99999836511
\end{pmatrix}
\]
Ответ очень близко расположен к единичному вектору СЛАУ, следовательно, метод работает.

\subsection{Случайная матрица}
На случайных, плохо обусловленных матрицах нельзя ожидать сходимость метода и корректной работы алгоритма. \\
После $n=30$ операций был получен результат;
\[
x_{30} = \randomResult
\]
Очевидно, итерационная последовательность расходится и этим методом нельзя найти решение такой СЛАУ.

\subsection{Матрица Гильберта}
Матрица Гильберта является очень плохо обусловленной и на ней так же ожидается расхождение итерационной последовательности. \\
Это подтверждается результатом на большом количеством итераций $n=1000$:
\[
x_{1000} = \hilbertResult
\]

\section{Исходный код}
  \code{lib.py}{python3}

\end{document}
