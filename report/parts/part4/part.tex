\documentclass[../../report.tex]{subfiles}

\begin{document}

\chapter[Метод Зайделя с последовательной релаксацией]{\texorpdfstring{Метод Зайделя \\ 
с последовательной релаксацией}{Метод Зайделя с последовательной релаксацией}}

\section{Описание}
Метод Зейделя с последовательной релаксацией это вариант метода Зейделя для решения линейных систем уравнений с подбором отимального коэфициента релаксации. Позволяет добиться  более быстрой сходимости. \\
Исходной матрицей был выбран вектор $x_0$ 
\[
x_0 = \begin{pmatrix}
    0 \\ 
    0 \\ 
    0 \\ 
    0 \\ 
    0 \\ 
    0 \\ 
    0 \\ 
    0 \\ 
    0 \\ 
    0 
\end{pmatrix}
\]

\subsection{Матрица с диагональным доминированием}
После проведения $n=30$ итераций, был получен результат
\[
x_{30} = \begin{pmatrix}
    0.15188234 \\ 
    0.10664535 \\
    0.06762958 \\ 
    0.11348456 \\
    0.16090425 \\
    0.18788446 \\
    0.08590699 \\
    0.06961082 \\
    0.10119528 \\
    0.05565383
\end{pmatrix}
\]

Такие значения переменных при умножении на матрицу дают следующий ответ
\[
\begin{pmatrix}
    1.00000011346 \\
    0.99999382372 \\
    0.99999457808 \\
    0.9999938682 \\
    0.99999288919 \\
    0.99999349964 \\
    0.9999946181 \\
    0.99999464458 \\
    0.99999768484 \\
    0.99999836511
\end{pmatrix}
\]
Ответ очень близко расположен к единичному вектору СЛАУ, следовательно, метод работает.

\subsection{Случайная матрица}
На случайных, плохо обусловленных матрицах нельзя ожидать сходимость метода и корректной работы алгоритма. \\
После $n=30$ операций был получен результат;
\[
x_{30} =
\begin{pmatrix}
    -1.06139262 * 10^{51} \\
     1.91636147 * 10^{51} \\ 
    -2.20198358 * 10^{51} \\ 
    -1.55514353 * 10^{50} \\
    -7.15823238 * 10^{51} \\ 
     5.45658686 * 10^{51} \\
    -6.04584125 * 10^{50} \\
     1.09411930 * 10^{52} \\
    -3.06084065 * 10^{52} \\
     3.40356171 * 10^{52}
\end{pmatrix}
\]
Очевидно, итерационная последовательность расходится и этим методом нельзя найти решение такой СЛАУ.

\subsection{Матрица Гильберта}
Матрица Гильберта является очень плохо обусловленной и на ней так же ожидается расхождение итерационной последовательности. \\
Это подтверждается результатом на большом количеством итераций $n=1000$:
\[
x_{1000} =
\begin{pmatrix}
     1.91416470 * 10^{44} \\
     1.46498715 * 10^{44} \\
    -8.17068353 * 10^{44} \\
     1.07334252 * 10^{45} \\
    -3.39269757 * 10^{45} \\
     4.73313389 * 10^{45} \\
    -8.02188440 * 10^{45} \\
     8.26239458 * 10^{45} \\
    -8.93384865 * 10^{45} \\
     7.76497623 * 10^{45}]
\end{pmatrix}
\]

\section{Исходный код}
  \code{lib.py}{python3}

\end{document}
