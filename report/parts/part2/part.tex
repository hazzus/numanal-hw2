\documentclass[../../report.tex]{subfiles}

\newcommand{\dominatedResult}{\begin{pmatrix}
    0.15188234 \\ 
    0.10664535 \\
    0.06762958 \\ 
    0.11348456 \\
    0.16090425 \\
    0.18788446 \\
    0.08590699 \\
    0.06961082 \\
    0.10119528 \\
    0.05565383
\end{pmatrix}}


\newcommand{\randomResult}{
\begin{pmatrix}
    -1.06139262 * 10^{51} \\
     1.91636147 * 10^{51} \\ 
    -2.20198358 * 10^{51} \\ 
    -1.55514353 * 10^{50} \\
    -7.15823238 * 10^{51} \\ 
     5.45658686 * 10^{51} \\
    -6.04584125 * 10^{50} \\
     1.09411930 * 10^{52} \\
    -3.06084065 * 10^{52} \\
     3.40356171 * 10^{52}
\end{pmatrix}
}

\newcommand{\hilbertResult}{
\begin{pmatrix}
     1.91416470 * 10^{44} \\
     1.46498715 * 10^{44} \\
    -8.17068353 * 10^{44} \\
     1.07334252 * 10^{45} \\
    -3.39269757 * 10^{45} \\
     4.73313389 * 10^{45} \\
    -8.02188440 * 10^{45} \\
     8.26239458 * 10^{45} \\
    -8.93384865 * 10^{45} \\
     7.76497623 * 10^{45}]
\end{pmatrix}
}


\begin{document}

\chapter{Метод простых итераций Якоби}

\section{Описание}
Метод простых итераций Якоби подходит для численного решения 
систем линейных уравнений на матрицах с диагональным доминированием.

Перед началом работы алгоритма уравнение $Ax=b$ приводится к виду $x = Dx + c$,
где $d_{i,j} = - \frac{a_{i,j}}{a_{i,i}}$, и $d_{i,i} = 0$. 
В свою очередь, $c_i = \frac{b_i}{a_{i,i}}$. Это позволяет применить метод простых итераций,
который использовался при поиске корня обычных уравнений.

В качестве начального приближения $x_0$ для всех матриц был принят 
\[
x_0 = 
\begin{pmatrix} 
  0 \\ 
  0 \\ 
  0 \\ 
  0 \\ 
  0 \\ 
  0 \\ 
  0 \\ 
  0 \\ 
  0 \\ 
  0 
\end{pmatrix}
\]

\subsection{Матрица с диагональным доминированием}
Запуск алгоритма на подобных матрицах приведёт к 
решению независимо от выбора начального приближения. Убедимся в этом:\\

Матрица $D$, которая участвует в итерационном вычислении, получила следующий вид: \\
\[
\begin{pmatrix}
0 & -0.190 & -0.087 & -0.037 & -0.139 & -0.202 & -0.072 & -0.136 & -0.009 & -0.006 \\
-0.207 & 0 & -0.030 & -0.032 & -0.062 & -0.131 & -0.022 & -0.132 & -0.099 & -0.135 \\
-0.140 & -0.015 & 0 & -0.143 & -0.093 & -0.094 & -0.032 & -0.086 & -0.058 & -0.095 \\
-0.108 & -0.100 & -0.124 & 0 & -0.076 & -0.074 & -0.075 & -0.073 & -0.187 & -0.019 \\ 
-0.055 & -0.055 & -0.199 & -0.000 & 0 & -0.071 & -0.223 & -0.110 & -0.104 & -0.121 \\
-0.088 & -0.012 & -0.093 & -0.015 & -0.003 & 0 & -0.189 & -0.105 & -0.115 & -0.120 \\
-0.161 & -0.119 & -0.135 & -0.094 & -0.113 & -0.064 & 0 & -0.037 & -0.077 & -0.008 \\
-0.067 & -0.025 & -0.146 & -0.158 & -0.091 & -0.147 & -0.041 & 0 & -0.129 & -0.143 \\
-0.131 & -0.188 & -0.104 & -0.180 & -0.004 & -0.017 & -0.119 & -0.078 & 0 & -0.042 \\
-0.099 & -0.074 & -0.144 & -0.028 & -0.128 & -0.097 & -0.020 & -0.127 & -0.067 & 0 
\end{pmatrix}
\] \\

Среди всех собственных значений матрицы $D^T D$ максимальное $\lambda_{max} = 0.73477$. 
Из этого можно вычислить норму матрицы $D$: $||D||_2 = 0.85718 \leq 1$. 
Всё это означает, что на данной матрице метод простых итераций Якоби сойдётся
независимо от начального приближения. Действительно:

После проведения $n = 84$ итераций, мы получили результат 
\[
x_{84} = \dominatedResult
\]
Такие значения переменных при умножении на матрицу дают следующий ответ
\[
\begin{pmatrix}
0.9999996660 \\
0.9999995911 \\
0.9999995245 \\
0.9999995973 \\
0.9999996311 \\
0.9999997107 \\
0.9999995615 \\
0.9999994838 \\
0.9999995505 \\
0.9999994698
\end{pmatrix}
\] \\
Этот ответ достаточно близко расположен к единичному вектору свободных членов СЛАУ,
поэтому можно сделать вывод, что метод Якоби работает корректно на таких матрицах.

В алгоритме использовался поправочный множитель $\frac{1 - ||D||_2}{||D||_2}$

\subsection{Случайная матрица}
Случайно выбранные матрицы в общем случае не имеют свойства диагонального доминирования, 
поэтому корректная работа алгоритма на них не ожидается.

При запуске алгоритма, норма разности между двумя соседними приближениями 
уже на девятой итерации стала больше допустимого значения $10^9$. \\
\[
x_{9} = \randomResult
\]
Итерационная последовательность явно разошлась.

\subsection{Матрица Гильберта}
На этой матрице ожидается сильное расхождение итерационной последовательности. 
Действительно, на десятой итерации норма малости превысила допустимое значение:
\[
x_{10} = \hilbertResult
\]
\section{Исходный код}
    \code{JacobiSystemSolver.cpp}{C++}
\end{document}
