\documentclass[../../report.tex]{subfiles}

\begin{document}

\chapter{Метод простых итераций Якоби}

\section{Описание}
Метод простых итераций Якоби подходит для численного решения 
систем линейных уравнений на матрицах с диагональным доминированием.

Перед началом работы алгоритма уравнение $Ax=b$ приводится к виду $x = Dx + c$,
где $d_{i,j} = - \frac{a_{i,j}}{a_{i,i}}$, и $d_{i,i} = 0$. 
В свою очередь, $c_i = \frac{b_i}{a_{i,i}}$. Это позволяет применить метод простых итераций,
который использовался при поиске корня обычных уравнений.

\subsection{Матрица с диагональным доминированием}
Запуск алгоритма на подобных матрицах приведёт к 
решению независимо от выбора начального приближения. \\
В качестве исходного вектора $x_0$ был выбран 
\[
x_0 = 
\begin{pmatrix} 
  0.335426 \\ 
  0.273642 \\ 
  0.951602 \\ 
  0.436853 \\ 
  0.406371 \\ 
  0.935238 \\ 
  0.431647 \\ 
  0.142821 \\ 
  0.769235 \\ 
  0.967797 
\end{pmatrix}
\]

После проведения $n = 30$ итераций, мы получили результат 
\[
x_{30} = 
\begin{pmatrix} 
0.154703 \\ 
0.109403 \\ 
0.070100 \\ 
0.116197 \\ 
0.163861 \\ 
0.190267 \\ 
0.088534 \\ 
0.072591 \\ 
0.103984 \\ 
0.058208  
\end{pmatrix}
\]
Такие значения переменных при умножении на матрицу дают следующий ответ
\[
\begin{pmatrix}
1.020008 \\
1.024496 \\
1.028487 \\
1.024122 \\
1.022096 \\
1.017332 \\
1.026267 \\
1.030923 \\
1.026925 \\
1.031761
\end{pmatrix}
\]
Этот ответ достаточно близко расположен к единичному вектору свободных членов СЛАУ,
поэтому можно сделать вывод, что метод Якоби работает корректно на таких матрицах.

\subsection{Случайная матрица}
Случайно выбранные матрицы в общем случае не имеют свойства диагонального доминирования, 
поэтому корректная работа алгоритма на них не ожидается.
\[
x_0 = 
\begin{pmatrix} 
0.847869 \\
0.974057 \\
0.879161 \\
0.79724 \\
0.0487656 \\
0.014166 \\
0.158691 \\
0.776082 \\
0.0784487 \\
0.383221 
\end{pmatrix}
x_{30} = 
\begin{pmatrix}
-691872645.608022 \\
-3666158086.439201 \\
-776755492.134045 \\
-1340839533.262990 \\
-1945873197.833104 \\
-646842902.245923 \\
-2526159091.968067 \\
-1027474353.002403 \\
-2233350018.232313 \\
-987202028.793846
\end{pmatrix}
\]
Итерационная последовательность явно разошлась.

\subsection{Матрица Гильберта}
На этой матрице ожидается сильное расхождение итерационной последовательности. Действительно:
\[
x_0 = 
\begin{pmatrix}
0.112103 \\
0.0558762 \\
0.4284 \\
0.53928 \\
0.809825 \\
0.974462 \\
0.49735 \\
0.758847 \\
0.774551 \\
0.557219
\end{pmatrix}
x_{30} = 
\begin{pmatrix}
-259588794.630372 \\
-431231576.367772 \\
-557710040.864852 \\
-658374381.972990 \\ 
-739652966.916937 \\
-809403993.098984 \\
-868479856.652358 \\
-920268668.283018 \\
-948626348.942869 \\
-993631397.152557
\end{pmatrix}
\]
\section{Исходный код}
    \code{JacobiSystemSolver.cpp}{C++}
\end{document}
